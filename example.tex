\documentclass[12pt, a4paper]{article}
\usepackage{ctex}
\usepackage{amsmath, amssymb}
\usepackage{geometry}
\geometry{left=2.5cm, right=2.5cm, top=2.5cm, bottom=2.5cm}

\title{一元二次方程 练习题}
\author{}
\date{}

\begin{document}
\maketitle

\section*{例题练习}

\begin{enumerate}
    \item 解方程:$x^2 - 5x + 6 = 0$
    
    \textbf{解:} 
    
    方法一(因式分解法):
    
    原方程可化为:$(x-2)(x-3)=0$
    
    所以:$x_1=2, x_2=3$
    
    \item 解方程:$x^2 - 4x + 4 = 0$
    
    \textbf{解:}
    
    这是一个完全平方式:
    
    $(x-2)^2 = 0$
    
    所以:$x_1=x_2=2$(重根)
    
    \item 解方程:$2x^2 + 5x - 3 = 0$
    
    \textbf{解:}
    
    使用求根公式:$x = \frac{-b \pm \sqrt{b^2-4ac}}{2a}$
    
    这里$a=2, b=5, c=-3$
    
    判别式:$\Delta = b^2 - 4ac = 25 - 4 \times 2 \times (-3) = 25 + 24 = 49$
    
    因此:$x = \frac{-5 \pm \sqrt{49}}{4} = \frac{-5 \pm 7}{4}$
    
    所以:$x_1 = \frac{1}{2}, x_2 = -3$
    
    \item 解方程:$x^2 - 2x - 8 = 0$
    
    \textbf{解:}
    
    因式分解:$(x-4)(x+2) = 0$
    
    所以:$x_1 = 4, x_2 = -2$
    
    \item 解方程:$3x^2 - 6x + 3 = 0$
    
    \textbf{解:}
    
    先提取公因数3:$3(x^2 - 2x + 1) = 0$
    
    即:$3(x-1)^2 = 0$
    
    所以:$x_1 = x_2 = 1$
    
    \item 解方程:$x^2 + x - 6 = 0$
    
    \textbf{解:}
    
    因式分解:$(x+3)(x-2) = 0$
    
    所以:$x_1 = -3, x_2 = 2$
    
    \item 解方程:$x^2 - 6x + 5 = 0$
    
    \textbf{解:}
    
    因式分解:$(x-1)(x-5) = 0$
    
    所以:$x_1 = 1, x_2 = 5$
    
    \item 解方程:$2x^2 - 7x + 3 = 0$
    
    \textbf{解:}
    
    因式分解:$(2x-1)(x-3) = 0$
    
    所以:$x_1 = \frac{1}{2}, x_2 = 3$
    
    \item 解方程:$x^2 + 4x + 3 = 0$
    
    \textbf{解:}
    
    因式分解:$(x+1)(x+3) = 0$
    
    所以:$x_1 = -1, x_2 = -3$
    
    \item 解方程:$x^2 - 9 = 0$
    
    \textbf{解:}
    
    这是平方差公式:$(x+3)(x-3) = 0$
    
    所以:$x_1 = 3, x_2 = -3$

\end{enumerate}

\section*{总结}

一元二次方程的解法主要有:
\begin{itemize}
    \item 因式分解法(最常用、最快捷)
    \item 配方法(理解二次函数的基础)
    \item 公式法(万能方法)
    \item 图像法(数形结合)
\end{itemize}

判别式$\Delta = b^2 - 4ac$决定了方程根的情况:
\begin{itemize}
    \item $\Delta > 0$:两个不相等的实数根
    \item $\Delta = 0$:两个相等的实数根(重根)
    \item $\Delta < 0$:无实数根(有两个共轭复数根)
\end{itemize}

\end{document}

